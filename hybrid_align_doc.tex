\documentclass[10pt]{article}
\addtolength{\textwidth}{3.0cm}
\addtolength{\hoffset}{-1.5cm}
%% \addtolength{\textheight}{4cm}
%% \addtolength{\voffset}{-2cm}
\usepackage{parskip}
\usepackage{draftwatermark}
\raggedright

\title{Genome / Transcriptome Hybrid Alignment}
\date{\today}
\pagestyle{empty}
\SetWatermarkText{SHOBDAR}
\SetWatermarkScale{4}
\SetWatermarkLightness{0.95}
\begin{document}

\maketitle

\section*{Overview}

This aligner attempts to provide a thorough alignment of fragments to
the genome and known transcriptome, assuming that the fragments arose
through simple fragmentation and end sequencing with a limited number
of sequencing errors. It does not attempt to provide alignments for
fragments to de novo junctions or fusions.

Because of this, the bowtie alignment steps for both genome and
transcriptome are run with the same settings for -I and -X (min and
max fragment length limits) as provided by the user. These settings
are meant to be real physical fragment lengths as estimated
independently. Also, for paired-end alignments, only cases in which
both reads are mapped in proper orientation, distance, and on the same
contig, are considered valid alignments.

\section*{Synopsis}

{\small
\begin{verbatim}
Produce inferred transcriptome and all ebwt indices

make -f align_prep.mk fa=hg19.fa gtf=GRCh37.60.filt.gtf


Produce the final alignment SAM file:

make -f hybrid_align.mk ebwt_dir=index read_tag=165_Mel \
reads1=165_Melanocytes_1.fastq reads2=165_Melanocytes_2.fastq \
fa=hg19.fa gtf=GRCh37.60.filt.gtf genome_sam_header=hg19.header.sam \
cal_file=top_stratum.qcal


Required arguments:

fa                 reference genome fasta file
gtf                transcript annotations
reads1             fastq first reads file
reads2             fastq second reads file
genome_sam_header  genome header file for chromosomes in fa
cal_file           mapq calibration file

Optional parameters (defaults in []):

ebwt_dir           directory for reference ebwt indices [<same as fa directory>]
ncpu               number of cpu passed to -p option of bowtie [12]
min_fraglen        minimum physical fragment length (a.k.a. `insert size' from library) [0]
max_fraglen        maximum physical fragment length [500]
bowtie_param1      first pass bowtie params [--chunkmbs 1024 --fr -l 28 -n 3 -e 200 -k 50]
bowtie_param2      second pass bowtie params [--chunkmbs 1024 --fr -y -l 24 -n 3 -e 400 -a]
read_tag           nickname for this reads dataset [`reads']
do_tx_pass2        set to `yes' or `no' if second bowtie pass on transcriptome is desired [`yes']
do_gen_pass2       set to `yes' or `no' if second bowtie pass on genome is desired [`no']
add_xs_tag         set to `-x' or blank to add (or not) cufflinks transcript sense tag `XS:i:' [`-x']
intron_cigar_n     `-n' gives spanned introns the CIGAR `N' operation. Otherwise, `D' is used [`-n']

Make targets:

all (or empty)     makes the full hybrid alignment file against genome
                   and separate alignment file against transcriptome
tx_single          subset of hybrid containing transcriptomic single mappers
tx_multi           subset containing transcriptomic multi mappers
genome_single      subset containing genomic-origin (see `Usage') single mappers
genome_multi       subset containing genomic-origin multi-mappers
unmapped           nonmappers
clean_sam          delete all sam files, but preserve ebwt indexes
\end{verbatim}
}

\section*{Workflow}

1. Align all fragments to transcriptome with first-pass bowtie
parameters.

2. Align unmapped fragments from (1) to transcriptome with second-pass
bowtie parameters.

3. Align all fragments to genome with first-pass bowtie parameters.

4. Align unmapped fragments from (3) to genome with second-pass bowtie
parameters.

5. Project alignments (1) and (2) to genome, reporting only one of
sets of identical fragment alignments.

6. Take union of (3), (4) and (5), report only one of sets of
identical fragment alignments.

7. Set mapq, primary alignment flag, and tags of entries in (6) for
further filtering.


\section*{Detail}

\subsection*{Steps 1-4}

These are an efficient way to get as thorough a set of alignments as
possible without trimming and without excessively long run times. The
stringencies of the first and second pass are necessarily tradeoffs
between speed and desired sensitivity. Typically, the first pass on
transcriptome places about 85 \% of the data and the second pass
places another 10-50\% of the remaining data. Depending on the
parameters, the second pass may take 5 times longer than the first
pass. It may be skipped if desired.

\subsection*{Step 5} 

This uses the transcriptome GTF to define the projections to the
genome. Note that transcriptome-to-genome projection is a many-to-one
reduction that destroys information about the implied fragment
size. During this operation, the first alignment in the set of
duplicate projected alignments is retained with its isize field
intact. All projected alignments are tagged XP:A:T.

\subsection*{Step 6}

The set of projected transcriptome alignments from step 5 are combined
with the set of direct genome alignments from steps 3 and
4. Individually, the set of projected transcriptome alignments are
distinct among themselves. Also, the set of genome alignments are
distinct among themselves. When combining them, there are only three
possibilities for duplicate alignments.

Possibility 1: If the genome alignment for a fragment is distinct, the
alignment is tagged XP:A:G.

Possibility 2: If the transcriptome projected to genome alignment is
distinct, it retains tag XP:A:T.

Possibility 3: If the genome and projected alignments are identical to
each other, only one is retained, and the record is tagged XP:A:M
(merged).



\subsection*{Step 7}

This step sets the mapq, primary alignment flag, and tags of the
merged de-duplicated entries in step 6. The logic is as follows. Each
fragment provides a list of alignments. Each alignment produces a raw
alignment score as follows: if both reads are mapped in proper
orientation, on the same contig, and within acceptable length range,
raw score equals the sum of mismatches of each individual read
alignment. Otherwise, the fragment `default missing' score, defined to
be one higher than the max valid fragment score is supplied.

Then the list of raw alignment scores for each fragment alignment are
grouped into strata. For example, alignment scores (0,2,5,2,0,3,10)
provide strata (0,2,3,5,10).

The mapping quality of a given alignment is determined from a
lookup-table stored in a .qcal file, based on the top and second
strata raw scores of that fragment, together with the given alignment
raw score. This format affords the user with finer control over which
raw score combinations should be ignored or downgraded, but may also
be used simply to assign high score to top stratum and low score to
lower strata.

The .qcal file has the following format:

{\small
\begin{verbatim}
score_tag: NM
max_valid_fragment_score: 30
larger_score_better: N
31      31     31      0
12      31     31      0
12      31     12    255
...
0       2       3      0
0       2       2      0
0       2       0    255
...
0       1       2      0
0       1       1      0
0       1       0    255
\end{verbatim}
}

The columns of the data are:

{\small
\begin{verbatim}
Column 1: top stratum raw score
Column 2: second stratum raw score
Column 3: given raw score
Column 4: assigned mapping quality
\end{verbatim}
}

For example, for the a set of fragment alignments with raw scores
(0,2,3,5,10), the top stratum score is 0, and the second stratum score
is 2. For alignments with a given score of zero, the relevant
calibration line is:

{\small \verb!0       2       0    255!}

and these alignments will receive a mapq of 255. Since these
accuracies are calculated on a stratum level, the given mapq's are
insensitive to stratum size. In short, this is desirable for
evaluating aligner accuracy since it reflects only the search
properties of the aligner rather than the redundancy level of the
reference. For the purposes of choosing single and multi-mappers in
the top stratum, a combination filter against mapq and tags is
provided as described below.

\section*{Usage}

For a given fragment, each alignment produces a raw score, which may
be stratified, meaning that the alignments are grouped into strata
having identical raw score.  Doing so produces a stratum rank and
stratum size. Usually, what one is interested in is the top-stratum
(rank = 1), and perhaps the subset of those that are single-mappers
(stratum size = 1).

However, in the context of hybrid alignment, there may be the case
where a score-based stratum contains both XP:A:T and XP:A:G
alignments. Even though both are in the same stratum, there are cases
where one prefers the XP:A:T alignment. A solution to this is to
stratify by raw score, breaking ties by transcript (XP:A:T, XP:A:M)
vs. genome (XP:A:G).

For example, suppose we had alignments (0M, 2G, 3T, 2G, 0G, 2M). Using
this scheme, they would be stratified as: (0T, 0G, 2T, 2G, 3T),
(symbolically, M is equivalent to T here). So the top stratum is 0T,
and even though there may be non-equivalent genomic alignments
attaining a raw score of zero, they fall in the second stratum.

Stratifying this way, we have four situations in the top stratum:

\begin{enumerate}

\item \textbf{Transcriptomic single mappers.} Tag combination `XP:A:[TM]
  XY:i:1 XZ:i:1'. In this most desirable situation, the best alignment
  is a single-mapping transcript alignment. Though there may be a
  genome alignment achieving the same score, it is by definition
  downgraded to a lower stratum and not considered as a serious
  alternative to this.

\item \textbf{Transcriptomic multi-mappers.}  Tag combination
  `XP:A:[TM] XY:i:1 XZ:i:[23...]' We are confident that the fragment
  came from transcript, but it is not possible to discern among a
  number of alternatives. These may be a mix of transcripts in the
  same gene, or among orthologs. Note: though many of these records
  are single exon and thus have tag XP:A:M (merged transcript and
  genome), the interpretation should be that they come from the
  transcriptome.

\item \textbf{Genomic single mappers.} Tag
  combination `XP:A:G XY:i:1 XZ:i:1'. This is a genome single
  mapper. There was no equal or better scoring alignment to the
  transcriptome. This is strong evidence that this fragment came from
  the genome.

\item \textbf{Genomic multi-mappers.} Tag
  combination `XP:A:G XY:i:1 XZ:i:[23...]'. As in (3) but now a genome
  multimapper. Unlike transcript multimappers, genome multimappers
  must be from orthologous pieces of DNA.

\item \textbf{Nonmappers.} These are candidates for
  de-novo junction discovery or fusions. They may also be the result
  of too many sequencing errors to map.

\end{enumerate}

All five of these categories can be output in separate SAM files if
desired, using make targets \textbf{tx\_single}, \textbf{tx\_multi},
\textbf{genome\_single}, \textbf{genome\_multi}, and
\textbf{unmapped}. The main result file is the default target and
includes all records.


\section*{Example Usage}

Usage:

{\small
\begin{verbatim}

[hbigelow@ussf-papp-gau02 align_input]$ make -n -f ~/makefiles/hybrid_align.mk \
     ebwt_dir=index bowtie_threads=26 read_tag=165_Melanocytes_all \
     reads1=165_Melanocytes_1.fastq reads2=165_Melanocytes_2.fastq \
     fa=hg19.fa gtf=GRCh37.60.filt.gtf genome_sam_header=hg19.header.sam cal_file=top_stratum.qcal

echo -e "$(date '+%F %T'): transcriptome alignment first pass"

bowtie --sam -q --un 165_Melanocytes_all.vs.hg19_to_GRCh37.60.filt.txome1.unmapped.fq \
     --chunkmbs 1024 --fr -l 28 -n 3 -e 200 -k 50 -I 0 -X 500 -p 12 \
	 index/hg19_to_GRCh37.60.filt \
     -1 165_Melanocytes_1.fastq -2 165_Melanocytes_2.fastq \
     165_Melanocytes_all.vs.hg19_to_GRCh37.60.filt.txome1.sam

echo -e "$(date '+%F %T'): genome alignment first pass"

bowtie --sam-nohead --sam -q --un 165_Melanocytes_all.vs.hg19.genome1.unmapped.fq \
     --chunkmbs 1024 --fr -l 28 -n 3 -e 200 -k 50 -I 0 -X 500 -p 12 index/hg19 \
     -1 165_Melanocytes_1.fastq -2 165_Melanocytes_2.fastq \
     165_Melanocytes_all.vs.hg19.genome1.sam

echo -e "$(date '+%F %T'): transcriptome alignment second pass"

bowtie --sam -q --un 165_Melanocytes_all.vs.hg19_to_GRCh37.60.filt.txome2.unmapped.fq \
     --chunkmbs 1024 --fr -y -l 24 -n 3 -e 400 -a -I 0 -X 500 -p 12 index/hg19_to_GRCh37.60.filt \
     -1 165_Melanocytes_all.vs.hg19_to_GRCh37.60.filt.txome1.unmapped_1.fq \
     -2 165_Melanocytes_all.vs.hg19_to_GRCh37.60.filt.txome1.unmapped_2.fq \
     165_Melanocytes_all.vs.hg19_to_GRCh37.60.filt.txome2.sam

cat 165_Melanocytes_all.vs.hg19_to_GRCh37.60.filt.txome1.sam > \
     165_Melanocytes_all.vs.hg19_to_GRCh37.60.filt.txome.sam
samtools view -S 165_Melanocytes_all.vs.hg19_to_GRCh37.60.filt.txome2.sam \
     >> 165_Melanocytes_all.vs.hg19_to_GRCh37.60.filt.txome.sam

echo -e "$(date '+%F %T'): sorting by alignment position"

align_eval sort -s ALIGN -m 4294967296 165_Melanocytes_all.vs.hg19_to_GRCh37.60.filt.txome.sam \
     165_Melanocytes_all.vs.hg19_to_GRCh37.60.filt.txome.asort.sam

echo -e "$(date '+%F %T'): samutil tx2genome ..."

samutil tx2genome -h hg19.header.sam hg19 GRCh37.60.filt.gtf \
     165_Melanocytes_all.vs.hg19_to_GRCh37.60.filt.txome.asort.sam \
     165_Melanocytes_all.vs.hg19.t2g.sam

echo -e "$(date '+%F %T'): genome alignment second pass"

bowtie --sam-nohead --sam -q --un 165_Melanocytes_all.vs.hg19.genome2.unmapped.fq \
     --chunkmbs 1024 --fr -y -l 24 -n 3 -e 400 -a -I 0 -X 500 -p 12 index/hg19 \
     -1 165_Melanocytes_all.vs.hg19.genome1.unmapped_1.fq \
     -2 165_Melanocytes_all.vs.hg19.genome1.unmapped_2.fq \
     165_Melanocytes_all.vs.hg19.genome2.sam

cat 165_Melanocytes_all.vs.hg19.genome1.sam 165_Melanocytes_all.vs.hg19.genome2.sam \
     > 165_Melanocytes_all.vs.hg19.genome.sam

echo -e "$(date '+%F %T'): merging alignments ..."

cat 165_Melanocytes_all.vs.hg19.t2g.sam 165_Melanocytes_all.vs.hg19.genome.sam \
     > 165_Melanocytes_all.vs.hg19.merged.sam

echo -e "$(date '+%F %T'): sorting by read id"

align_eval sort -s READ_ID_FLAG -m 4294967296 165_Melanocytes_all.vs.hg19.merged.sam \
     165_Melanocytes_all.vs.hg19.merged.rsort.sam

echo -e "$(date '+%F %T'): merging duplicate genome-projected and genome alignments  ..."

samutil merge_tg 165_Melanocytes_all.vs.hg19.merged.rsort.sam 165_Melanocytes_all.vs.hg19.tgmerged.sam

echo -e "$(date '+%F %T'): sorting by read id"

align_eval sort -s READ_ID_FLAG -m 4294967296 165_Melanocytes_all.vs.hg19.tgmerged.sam \
     165_Melanocytes_all.vs.hg19.tgmerged.rsort.sam

echo -e "$(date '+%F %T'): re-scoring mapq in final alignment"

samutil score_mapq -l 0 -L 500  top_stratum.qcal 165_Melanocytes_all.vs.hg19.tgmerged.rsort.sam \
     165_Melanocytes_all.vs.hg19.hybrid.sam

echo -e "$(date '+%F %T'): sorting by read id"

align_eval sort -s READ_ID_FLAG -m 4294967296 \
     165_Melanocytes_all.vs.hg19_to_GRCh37.60.filt.txome.sam \
     165_Melanocytes_all.vs.hg19_to_GRCh37.60.filt.txome.rsort.sam

echo -e "$(date '+%F %T'): re-scoring mapq in final alignment"

samutil score_mapq -l 0 -L 500  top_stratum.qcal \
     165_Melanocytes_all.vs.hg19_to_GRCh37.60.filt.txome.rsort.sam \
     165_Melanocytes_all.vs.hg19_to_GRCh37.60.filt.tx.sam

rm 165_Melanocytes_all.vs.hg19_to_GRCh37.60.filt.txome.rsort.sam \
     165_Melanocytes_all.vs.hg19_to_GRCh37.60.filt.txome2.sam \
     165_Melanocytes_all.vs.hg19.tgmerged.rsort.sam \
     165_Melanocytes_all.vs.hg19.t2g.sam \
     165_Melanocytes_all.vs.hg19.genome.sam \
     165_Melanocytes_all.vs.hg19.genome1.sam \
     165_Melanocytes_all.vs.hg19.genome2.sam \
     165_Melanocytes_all.vs.hg19_to_GRCh37.60.filt.txome.sam \
     165_Melanocytes_all.vs.hg19.merged.sam \
     165_Melanocytes_all.vs.hg19_to_GRCh37.60.filt.txome.asort.sam \
     165_Melanocytes_all.vs.hg19_to_GRCh37.60.filt.txome1.sam \
     165_Melanocytes_all.vs.hg19.tgmerged.sam
\end{verbatim}
}

\end{document}
